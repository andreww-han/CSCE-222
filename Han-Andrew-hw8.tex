\documentclass{article}
\usepackage{amsmath,amssymb,amsthm,latexsym,paralist}

\theoremstyle{definition}
\newtheorem{problem}{Problem}
\newtheorem*{solution}{Solution}
\newtheorem*{resources}{Resources}

\newcommand{\name}[2]{\noindent\textbf{Name: #1}\hfill \textbf{UIN: #2}}
\newcommand{\honor}{\noindent On my honor, as an Aggie, I have neither
  given nor received any unauthorized aid on any portion of the
  academic work included in this assignment. Furthermore, I have
  disclosed all resources (people, books, web sites, etc.) that have
  been used to prepare this homework. \\[2ex]
 \textbf{Electronic Signature: \underline{ Andrew Han } } }
 
\newcommand{\checklist}{\noindent\textbf{Checklist:}
\begin{compactitem}[$\Box$] 
\item Did you type in your name and UIN? 
\item Did you disclose all resources that you have used? \\
(This includes all people, books, websites, etc.\ that you have consulted.)
\item Did you electronically sign that you followed the Aggie Honor Code? 
\item Did you solve all problems? 
\item Did you submit both of the .tex and .pdf files of your homework to the correct link 
on eCampus? 
\end{compactitem}
}

\newcommand{\problemset}[1]{\begin{center}\textbf{Problem Set #1}\end{center}}
\newcommand{\duedate}[1]{\begin{quote}\textbf{Due dates:} Electronic
    submission of \textsl{yourLastName-yourFirstName-hw8.tex} and 
    \textsl{yourLastName-yourFirstName-hw8.pdf} files of this homework is due on
    \textbf{#1} on \texttt{http://ecampus.tamu.edu}. You will see two separate links
    to turn in the .tex file and the .pdf file separately. Please do not archive or compress the files.  
    \textbf{If any of the two submissions are missing, you will likely receive zero points for this 
    homework.}\end{quote} }

\newcommand{\N}{\mathbf{N}}
\newcommand{\R}{\mathbf{R}}
\newcommand{\Z}{\mathbf{Z}}


\begin{document}
\vspace*{-15mm}
\begin{center}
{\large
CSCE 222 [Sections 503, 504] Discrete Structures for Computing\\[.5ex]
Fall 2019 -- Hyunyoung Lee\\}
\end{center}
\problemset{8}
\duedate{Friday, 11/8/2019 before 10:00 p.m.}
\name{ Andrew Han }{ 227009495 }
\begin{resources} Discrete Mathematics and its Applications 8th Edition
\end{resources}
\honor

\bigskip
\noindent
Total 100 points.

\begin{problem} (2.5 points $\times$ 4 = 10 points)
Section 6.3, Exercise 20, page 435
\end{problem}
\begin{solution}
\textbf{a)} $ 10 \choose 3 $  = 120 \\
\textbf{b)} At least 6 0s. $ 10 \choose 6$ + $ 10 \choose 7$ + $10 \choose 8$ + $10 \choose 9$ + $10 \choose 10$ = 386 \\
\textbf{c)} At least seven 1s is the same as at most 3 0s. $10 \choose 0$ + $10 \choose 1$ + $10 \choose 2$ + $10 \choose 3$ = 176 \\
\textbf{d)} At least three 1s is the same as the total combinations subtracted by the strings with at most 2 1s. $2^{10} = 1024$ - $10 \choose 0$ - $10 \choose 1$ - $10 \choose 2$ = 968
\end{solution}

\begin{problem} (2 points $\times$ 5 = 10 points)
Section 6.3, Exercise 22 b), c), d), e), and f), page 435
\end{problem}
\begin{solution} 
\textbf{b)} $6! = 720$ \\ 
\textbf{c)} $5! = 120$ \\ 
\textbf{d)} $5! = 120$ \\ 
\textbf{e)} $4! = 24$ \\ 
\textbf{f)} Not possible to have both. If BCA exists, then ABF cannot exist and vice versa.  
\end{solution}

\begin{problem} (5 points $\times$ 2 = 10 points)
Section 6.4, Exercise 12 a) and b), page 444
\end{problem}
\begin{solution} 
\textbf{a)} $6 \choose 3$ * $(5x^{2})^{(6-3)}$ * $(2y^{3})^{3}$ \\ $6 \choose 3$ * $(5^{3})(2^{3}) x^{6}y^{9}$ \\ Coefficient is $20 * 125 * 8 = 20000$ \\ 
\textbf{b)} $ 6 \choose 5$ * $(5x^{2})^{(6-5)}$ * $(2y^{3})^{5}$ \\ $6 \choose 5$ * $(5^{1})(2^{5}) x^{2}y^{15}$ \\ Coefficient is $6 * 5 * 32 = 960$
\end{solution} 

\begin{problem} ($10+3+7=20$ points)
Section 8.1, Exercise 10, page 537. For a) and c), explain and show your work.
\end{problem}
\begin{solution}
\textbf{a)} Looking at a bit string a of length n, if you look at one number, the total number of bit strings containing the string 01 is going to be $a_{n-1}$, which we should already know through recursion. Now we can look at two numbers and the remaining bit string is going to be length of n-2. Looking at three, the remaining bit string is going to be length of n-3 and so on. Using this info, you can make a geometric sequence as follows. $$ \sum_{k = 0}^{n-2} 2^{k} $$ By using the definition of the sum of a geometric sequence, we can write this as $$ 2^{n-1} -1 $$ To get our closed form for $a_{n}$, we need to go back and add the $a_{n-1}$ that was found recursively  to get $$ a_{n} = a_{n-1} + 2^{n-1} -1$$ \\
\textbf{b)} A string needs to have at least a length of 2 to contain 01. That means the initial conditions are $a_{0} =0$ and $a_{1} = 0$ \\
\textbf{c)} $$ a_{0} = 0 $$ $$a_{1} = 0 $$ $$ a_{2} = a_{1} +2^1 - 1 = 1$$ $$ a_{3} = a_{2} +2^2 - 1 = 4$$ 
$$ a_{4} = a_{3} +2^3 - 1 = 11$$ $$ a_{5} = a_{4} +2^4 - 1 = 26$$ $$ a_{6} = a_{5} +2^5 - 1 = 57$$ $$ a_{7} = a_{6} +2^6 - 1 = 120$$ 

\end{solution}

\begin{problem} (20 points)
Section 8.1, Exercise 28, page 538. This problem has two parts as below.
\end{problem}
\begin{solution} 
\ \\
a) (10 points) \textit{Show that the Fibonacci numbers satisfy \ldots}\\[2ex] \\ The Fibonacci sequence is defined as $f_{n} = f_{n-1} + f_{n-2}$ We can expand this out to see if the recurrence relation given will satisfy this. $$f_{n} = f_{n-1} + f_{n-2} $$ $$ = (f_{n-2}+f_{n-3}) + (f_{n-3}+f_{n-4}) = f_{n-2} + 2f_{n-3} + f_{n-4}$$ $$ = (f_{n-3} + f_{n-4}) + 2*(f_{n-4}+f_{n-5}) + f_{n-4} = f_{n-3} + 4f_{n-4} + 2f_{n-5} $$ $$= f_{n-4} + f_{n-5} + 4f_{n-4} + 2f_{n-5} = 5f_{n-4} + 3f_{n-5} $$ 
b) (10 points) \textit{Use this recurrence relation to show that \ldots} (Prove by induction on $n$.) \\ Induction base: \\ P(1): $f_{5} = 5$ . Since 5 is divisble by 5, so P(1) holds. \\ Induction Step: As induction hypothesis, assume that P(n) holds. Then show that P(n+1) holds. $$f_{5(n+1)} = f_{5n} + f_{5n-1} \text{ by def. of Fibonacci numbers} $$ $$= 5f_{5(n-4)} + 3f_{5(n-5)} + 5f_{5((n-1)-4)} + 3f_{5((n-1)-5} \text{ by IH}$$ Because the two terms that can be formed from the expression above can be divided by 5 from the induction hypothesis, adding the two together will still result in a multiple of 5. Therefore, the claim holds by induction on n.
\end{solution}

\goodbreak
\noindent
\textit{For Problems 6 and 7, use Table 1 on page 568.}

\begin{problem} (5 points $\times$ 3 = 15 points)
Section 8.4, Exercise 6 b)--d), page 575
\end{problem}
\begin{solution} 
\textbf{b)} $\frac{2x}{1-2x}$ \\
\textbf{c)} $\frac{2x-1}{(1-x)^{2}}$ \\
\textbf{d)} $\frac{e^{x}-1}{x}$
\end{solution}

\begin{problem} (5 points $\times$ 3 = 15 points)
Section 8.4, Exercise 8 b)--d), page 575.
\end{problem}
\begin{solution} 
\textbf{b)} $3 \choose n$ * $-(-3)^{n}$ \\ 
\textbf{c)} If n is an even integer, $2^{n}$. If n is an odd integer, 0. \\
\textbf{d)} $\frac{n(n-1)}{2}$ for $n \geq 2$ and initial conditions $ a_{0} = 0$ $a_{1} = 0$ 
\end{solution}

\bigskip

\goodbreak
\checklist
\end{document}
