\documentclass{article}
\usepackage{amsmath,amssymb,amsthm,latexsym,paralist}

\theoremstyle{definition}
\newtheorem{problem}{Problem}
\newtheorem*{solution}{Solution}
\newtheorem*{resources}{Resources}

\newcommand{\name}[2]{\noindent\textbf{Name: #1}\hfill \textbf{UIN: #2}}
\newcommand{\honor}{\noindent On my honor, as an Aggie, I have neither
  given nor received any unauthorized aid on any portion of the
  academic work included in this assignment. Furthermore, I have
  disclosed all resources (people, books, web sites, etc.) that have
  been used to prepare this homework. \\[2ex]
 \textbf{Electronic Signature: \underline{ Andrew Han } } }
 
\newcommand{\checklist}{\noindent\textbf{Checklist:}
\begin{compactitem}[$\Box$] 
\item Did you type in your name and UIN? 
\item Did you disclose all resources that you have used? \\
(This includes all people, books, websites, etc.\ that you have consulted.)
\item Did you electronically sign that you followed the Aggie Honor Code? 
\item Did you solve all problems? 
\item Did you submit both of the .tex and .pdf files of your homework to the correct link 
on eCampus? 
\end{compactitem}
}

\newcommand{\problemset}[1]{\begin{center}\textbf{Problem Set #1}\end{center}}
\newcommand{\duedate}[1]{\begin{quote}\textbf{Due dates:} Electronic
    submission of \textsl{yourLastName-yourFirstName-hw9.tex} and 
    \textsl{yourLastName-yourFirstName-hw9.pdf} files of this homework is due on
    \textbf{#1} on \texttt{http://ecampus.tamu.edu}. You will see two separate links
    to turn in the .tex file and the .pdf file separately. Please do not archive or compress the files.  
    \textbf{If any of the two submissions are missing, you will likely receive zero points for this 
    homework.}\end{quote} }

\newcommand{\N}{\mathbf{N}}
\newcommand{\R}{\mathbf{R}}
\newcommand{\Z}{\mathbf{Z}}


\begin{document}
\vspace*{-15mm}
\begin{center}
{\large
CSCE 222 [Sections 503, 504] Discrete Structures for Computing\\[.5ex]
Fall 2019 -- Hyunyoung Lee\\}
\end{center}
\problemset{9}
\duedate{Friday, 11/15/2019 before 10:00 p.m.}
\name{ Andrew Han }{ 227009495 }
\begin{resources} Discrete Mathematics and its Applications 8th editon
\end{resources}
\honor

\bigskip
\noindent
Total 100 + 10 (extra credit) points.

\begin{problem} (10 pts $\times$ 3 = 30 points)
Section 8.2, Exercise 4 b), c) and d) page 551.  For each subproblem, find the 
closed form solution for $a_n$ by answering the following step by step:
\begin{enumerate}
\item (2 points) What is the characteristic equation of the recurrence relation?
\item ($2+2$ points) What are the roots of the characteristic equation?  
Express $a_n$ in a generic form in terms of the roots you found.  
For (d), you will get only one root; refer Theorem~2 in page~544 
for the generic form in this case.
\item (4 points) Find the closed form solution for $a_n$ using the initial conditions. 
Show your work.
\end{enumerate}
\end{problem}
\begin{solution} 
\textbf{b)} 1. 
\end{solution}

\begin{problem} (5 points $\times$ 3 = 15 points)
Let $A=\{1,2,3,4,5\}$ and $B=\{0,1,2,3,4\}$.
\begin{enumerate}[a)] 
\item List all the ordered pairs in the relation $R=\{(a,b)\mid a+b=5 \}$ on $A$.
\item List all the ordered pairs in the relation $R=\{(a,b)\mid a<b \}$ on $A$.
\item List all the ordered pairs in the relation $R=\{(a,b)\mid a<b \}$ from $A$ to $B$.
\end{enumerate}
\end{problem}
\begin{solution} 
\end{solution}

\begin{problem} (8 points $\times$ 3 = 24 points)
Section 9.1, Exercise 6 a), b) and d), page 608 
\end{problem}
\begin{solution} 
\end{solution}

\begin{problem} (5 points $\times$ 2 = 10 points)
Let $A$ be the set of all people and $(x,y)\in A\times A$.  Is each of the 
following an equivalence relation?  For each subproblem, explain which 
of the three properties of an equivalence relation -- reflexivity, symmetry, 
and transitivity -- are satisfied and which are not, by explaining why or why not.
Then, answer whether it is an equivalence relation.
\begin{enumerate}[a)]
\item $R_1 = \{(x,y) \mid x \text{ and } y \text{ have the same parents}\}$
\item $R_2 = \{(x,y) \mid x \text{ and } y \text{ have a common grandparent}\}$
\end{enumerate}
\end{problem}
\begin{solution} 
\textbf{a)} Reflexive because if $x,x \in \R$, then x will have the same parents as x. \\ Symmetric because if $x,y \in \R$ and $y,x \in \R$, x has the same parents as y and y has the same parents as x. \\ Transitive because if $x,y \in \R$ and x has the same parents as y and $y,z \in \R$ amd y has the same parents as z, then $x,z \in \R$ and x will have the same parents as z. All three properties are satisfied, so this is an equivalence relation. \\
\textbf{b)} Reflexive because if $x,x \in \R$ then x will have a common grandparent with x. \\ Symmetric because if $x,y \in \R$ and x and y have a common grandparent, then if $y,x \in \R$, y and x will have a common grandparent. \\ Not transitive because if $x,y \in \R$ and x and y have a common grandparent, an $y,z \in \R$ and y and z have a common grandparent, x and z don't necessraily have a common grandparent. Because not all three properties are met, this is not an equivalence relation. 
\end{solution}

\begin{problem} (10 points)
We define on the set $\N_1=\{1,2,3,\cdots\}$ of positive integers a relation 
$\sim$ such that two positive integers $x$ and $y$ satisfy $x\sim y$ if and only 
if $x/y=2^k$ for some integer $k$. Show that $\sim$ is an equivalence relation.
\end{problem}
\begin{solution} 
Reflexive because for $x,x \in \N_{1}, x/x = 1 = 2^{k} = 2^0$. \\ Symmetric because for $x,y \in \N_{1}, \frac{x}{y} = 2^{k}$ and $y,x \in \N_{1} = \frac{y}{n} = 2^{-k}$ because $\frac{y}{x}$ is the inverse of $\frac{x}{y}$ \\ Transitive because if $x,y \in \N_{1}$ and $y,z \in \N_{1}$, then $\frac{x}{y} = 2^k$ and $\frac{y}{z} = 2^{l}$. Then for $x,z \in \N_{1}, \frac{x}{z} = \frac{x}{y} * \frac{y}{z} = 2^{k}*2{l} = 2^{k+l}$ \\ Therefore, $\sim$ is an equivalence relation.
\end{solution}

\begin{problem} (7 points $\times$ 3 = 21 points)
Section 9.6, Exercise 2 b), c) and e), page 662.  
For each subproblem, explain which of the three properties of a partial ordering 
-- reflexivity, antisymmetry, and transitivity -- are satisfied and which are not, 
by explaining why or why not.  Then, answer whether it is a partial ordering.
\end{problem}
\begin{solution} 
\textbf{b)} Reflexive because for each element, a, there exists a pair on the relation where $a,a \in \R$ \\ Anti-symmetric because for $2,0 \in \R, 2,3 \in \R$, the condition that $a,b \in \R \land b,a \in \R$ is false, meaning the whole statement is vacuously true. The rest of the pairs satisfies the condition $a,b \in \R \land b,a \in \R$ and implies $a=b$ \\ Transitive because for all the pairs where $a,b \in \R$ and $b,c \in \R, a,c \in \R$ also exists in the relation. Thus, the relation has all the properties of a partial ordering. \\ 
\textbf{c)}  Reflexive because for each element, a, there exists a pair on the relation where $a,a \in \R$ \\ Anti-symmetric because for the two pairs, (1,2) and (3,1), $a,b \in \R$ exists but $b,a \in \R$ does not hold true, so the claim holds vacuously true. For all the other pairs, both $a,b \in \R$ and $\b,a \in \R$ hold true, making the claim true. \\ Not transitive because $3,1 \in \R$ and   $1,2 \in \R$ exist, but $3,2 \in \R$ is not true. Thus,  the relation does not have all the properties of a partial ordering. 
\end{solution}

\goodbreak
\checklist
\end{document}
