\documentclass{article}
\usepackage{amsmath,amssymb,amsthm,latexsym,paralist}

\theoremstyle{definition}
\newtheorem{problem}{Problem}
\newtheorem*{solution}{Solution}
\newtheorem*{resources}{Resources}

\newcommand{\name}[2]{\noindent\textbf{Name: #1}\hfill \textbf{UIN: #2}}
\newcommand{\honor}{\noindent On my honor, as an Aggie, I have neither
  given nor received any unauthorized aid on any portion of the
  academic work included in this assignment. Furthermore, I have
  disclosed all resources (people, books, web sites, etc.) that have
  been used to prepare this homework. \\[2ex]
 \textbf{Electronic Signature: \underline{ Andrew Han } } }
 
\newcommand{\checklist}{\noindent\textbf{Checklist:}
\begin{compactitem}[$\Box$] 
\item Did you type in your name and UIN? 
\item Did you disclose all resources that you have used? \\
(This includes all people, books, websites, etc.\ that you have consulted)
\item Did you electronically sign that you followed the Aggie honor code? 
\item Did you solve all problems? 
\item Did you submit both of the .tex and .pdf files of your homework separately 
to the correct link on eCampus? 
\end{compactitem}
}

\newcommand{\problemset}[1]{\begin{center}\textbf{Problem Set #1}\end{center}}
\newcommand{\duedate}[1]{\begin{quote}\textbf{Due dates:} Electronic
    submission of \textsl{yourLastName-yourFirstName-hw10.tex} and 
    \textsl{yourLastName-yourFirstName-hw10.pdf} files of this homework is due on
    \textbf{#1} on \texttt{https://ecampus.tamu.edu}. You will see two separate links
    to turn in the .tex file and the .pdf file separately. Please do not archive or compress the files.  
    \textbf{If any of the two submissions are missing, you will likely receive zero points for this 
    homework.}
    \end{quote} }

\newcommand{\N}{\mathbf{N}}
\newcommand{\R}{\mathbf{R}}
\newcommand{\Z}{\mathbf{Z}}


\begin{document}
\vspace*{-20mm}
\begin{center}
{\large
CSCE 222 [Sections 503, 504] Discrete Structures for Computing\\[.5ex]
Fall 2019 -- Hyunyoung Lee\\}
\end{center}
\problemset{10}
\duedate{Wednesday, 11/27/2019 before 10:00 p.m.}
\name{ Andrew Han }{ 227009495 }
\begin{resources} Discrete Mathematics and its Applications 8th Edition
\end{resources}
\honor

\bigskip
\noindent
In this problem set, you can earn up to $100+10$ (extra credit) points.

\begin{problem} ($5+5+10=20$ points)
Section 13.1, Exercise 4, page 894
\end{problem}
\begin{solution} 
\textbf{a)} $S \rightarrow 1S, 1S \rightarrow 11S, 11S \rightarrow 111S, 111S \rightarrow 11100A, 11100A \rightarrow 111000$ \\
\textbf{b)} 11001 does not belong to the language generated by G because once the production sequence no longer applies the $S \rightarrow 1S$ rule and begins to apply the rules with A involved, it is no longer possible to produce another S, which also means another 1 cannot be produced. Looking at 11001, after the first two 1s in the set, the set starts to produce 0s, which means that it is not possible for that last number to be a 1. \\
\textbf{c)} $L(G) = \{1^{a}0^{b} | a \geq 0, b \geq 3\}$, 1 can continually be added to the set as long as P recursively calls 1S. 0 will continuall be added to the set as long as P recursively calls 0A. b has to be $\geq$ 3 because of the rule $S \rightarrow 00A$. The minimum amount of 0s possible is 3.
\end{solution}

\begin{problem} (10 points)
Section 13.1, Exercise 6 d), page 894
\end{problem}
\begin{solution} 
$L(G) = \{a^{2n} | n \geq 2\}$ OR $L(G) = \{b^n | n \geq 1\}$
\end{solution}

\begin{problem} (10 points)
Section 13.1, Exercise 14 b), page 894
\end{problem}
\begin{solution} 
$G = \{V,T,S,P\}$, $V = \{0,1,S,A\}$, $T=\{0,1\}$, S = starting symbol, $P = \{S \rightarrow 00A, A \rightarrow AA, A \rightarrow 1\}$
\end{solution}

\begin{problem} (15 points $\times$ 2 = 30 points)
Consider the grammar $G = (V, T, E, P)$ for expressions ($E$ for short) such that 
$V = \{E, a, +, *, (, ) \}, T = \{a, +, *, (, ) \}, E$ is the starting symbol, and 
$$P = \{ E \rightarrow ( E ) \mid E + E \mid E * E \mid a \}.$$  
\begin{enumerate}[a)]
\item Explain whether $G$ is regular, context-free, or context-sensitive, respectively.
Explain why or why not.
\begin{solution} 
G is not regular grammar because the production rules do not satisfy the rule of the starting symbol producing the empty string, or a non-terminal symbol producing either a terminal symbol followed by a non-terminal symbol, or a non-terminal symbol producing a terminal symbol. \\ \\
G is context-free grammar because the production rules can be applied anywhere E is found no matter where it is in the string. In other words, it follows the rule of $A \rightarrow B$ where A is a non-terminal symbol and B is contained in $V^{*}$ \\ \\ 
Because G is context-free, it must also be context-sensitive because of the Chomsky heirarchy. 
\end{solution}
\item Explain the language $L(G)$ that is generated by $G$, especially, what kind of 
strings belong to the language. Be specific.  
Also, give five shortest strings that belong to $L(G)$.
\begin{solution} 
L(G) that is generated will contain "a" occurring at least 1 time, "+" and "*" occurring any positve amount of times, and "(" occurring the same amount of times as ")" does. The five shortest strings are: $$a, (a), a+a, a*a, (a+a) $$
\end{solution}
\end{enumerate}
\end{problem}

\begin{problem} (5 points $\times$ 2 = 10 points)
Section 13.2, Exercise 4 a) and b), page 902.  \textit{Explain} by showing the state 
transition and the output of each state.
\end{problem}
\begin{solution} 
\textbf{a)} Input 1 to $S_{0}$ produces output 0 and $S_{0} \rightarrow S_{2}$ \\  Input 0 to $S_{2}$ produces output 0 and $S_{2} \rightarrow S_{3}$ \\  Input 0 to $S_{3}$ produces output 1 and $S_{3} \rightarrow S_{1}$ \\  Input 0 to $S_{1}$ produces output 1 and $S_{1} \rightarrow S_{0}$ \\  Input 1 to $S_{0}$ produces output 0 and $S_{0} \rightarrow S_{2}$ \\ Output is 00110. \\
\textbf{b)}  Input 1 to $S_{0}$ produces output 1 and $S_{0} \rightarrow S_{2}$ \\  Input 0 to $S_{2}$ produces output 1 and $S_{2} \rightarrow S_{2}$ \\  Input 0 to $S_{2}$ produces output 1 and $S_{2} \rightarrow S_{2}$ \\ Input 0 to $S_{2}$ produces output 1 and $S_{2} \rightarrow S_{2}$ \\ Input 1 to $S_{2}$ produces output 0 and $S_{2} \rightarrow S_{0}$ \\ Output is 11110.
\end{solution}

\begin{problem} (10 points $\times$ 2 = 20 points)
Section 13.3, Exercise 8 e) and f), page 914.  \textit{Prove or disprove.}
\end{problem}
\begin{solution} 
\textbf{e)} $A^{*}A = A^{*}$ does not hold because A can be a non-empty string, meaning that the concatenation on the left-hand side would produce a result that longer has the empty string in it, making it not possible to be $A^{*}$ \\
\textbf{f)} This can be disproved using the subset $A = \{1,11\}$. $A^{2} = \{11,111,1111\}$ which means $|A^{2}| = 3$. However, $|A|^{2} = 4$ meaning that the claim $|A^{n}| = |A|^{n}$ does not hold. 
\end{solution}

\begin{problem} (5 points $\times$ 2 = 10 points)
Section 13.3, Exercise 10 b) and c) page 914.  \textit{Explain.}
\end{problem}
\begin{solution} 
\textbf{b)} The string 01001 is not in this set because the set is in the form of $\{0^{m}011^{n} | m,n \geq 0\}$. There cannot be another 0 after "01" occurs.\\
\textbf{c)} The string 01001 is in this set because the set is in the form of $\{(010)^{m}0^{n}1 | m,n \geq 0\}$ So if 010 and 0 occur one time, this string will be found in the set. 
\end{solution}

\goodbreak
\checklist
\end{document}
