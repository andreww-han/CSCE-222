\documentclass{article}
\usepackage{amsmath,amssymb,amsthm,latexsym,paralist}

\theoremstyle{definition}
\newtheorem{problem}{Problem}
\newtheorem*{solution}{Solution}
\newtheorem*{resources}{Resources}

\newcommand{\name}[2]{\noindent\textbf{Name: Andrew Han}\hfill \textbf{Section: 504}}
\newcommand{\honor}{\noindent On my honor, as an Aggie, I have neither
  given nor received any unauthorized aid on any portion of the
  academic work included in this assignment. Furthermore, I have
  disclosed all resources (people, books, web sites, etc.) that have
  been used to prepare this homework. \\[1.5ex]
 \textbf{Electronic Signature: \underline{ Andrew Han } } }
 
\newcommand{\checklist}{\noindent\textbf{Checklist:}
\begin{compactitem}[$\Box$] 
\item Did you type in your name and section? 
\item Did you disclose all resources that you have used? \\
(This includes all people, books, websites, etc.\ that you have consulted.)
\item Did you electronically sign that you followed the Aggie Honor Code? 
\item Did you try to solve all problems? 
\item Did you submit both of the .tex and .pdf files of your homework to the correct link on eCampus?
\end{compactitem}
}

\newcommand{\problemset}[1]{\begin{center}\textbf{Problem Set #1}\end{center}}
\newcommand{\duedate}[1]{\begin{quote}\textbf{Due dates:} Electronic
    submission of \textsl{yourLastName-yourFirstName-hw3.tex} and 
    \textsl{yourLastName-yourFirstName-hw3.pdf} files of this homework is due on
    \textbf{#1} on \texttt{http://ecampus.tamu.edu}. You will see two separate links
    to turn in the .tex file and the .pdf file separately. Please do not archive or compress the files.  
    \textbf{If any of the two submissions are missing, you will likely receive zero points for this 
    homework.} \end{quote} }

\newcommand{\N}{\mathbf{N}}
\newcommand{\R}{\mathbf{R}}
\newcommand{\Z}{\mathbf{Z}}


\begin{document}
\vspace*{-15mm}
\begin{center}
{\large
CSCE 222 [Sections 503, 504] Discrete Structures for Computing\\[.5ex]
Fall 2019 -- Hyunyoung Lee\\}
\end{center}
\problemset{3}
\duedate{Friday, 9/20/2019 before 10:00 p.m.}
\name{ Andrew Han }{ 504 }
\begin{resources} Discrete Mathematics and its Applications 8th Edition
\end{resources}
\honor

\bigskip
\noindent
Total: 100 (+ 5 extra) points

\medskip

\noindent
\textbf{*** Please make sure that you are solving the correct problems from 
the \underline{8th Edition of the Rosen book}, not the 7th Edition! ***}

\medskip

\begin{problem} (2 points $\times$ 6 subproblems = 12 points)
Section 2.1, Exercise 10, page 132.
\end{problem} 
\begin{solution}
\textbf{a)} No. This set does not contain any sets, so \{2\} is not an element of that set.
\\*
\textbf{b)} No. This set does not contain any sets, so \{2\} is not an element of that set.
\\*
\textbf{c)} Yes. One of the elements in this set is the set \{2\}. 
\\*
\textbf{d)} Yes. One of the elements in this set is the set \{2\}. 
\\*
\textbf{e)} Yes. One of the elements in this set is the set \{2\}. 
\\*
\textbf{f)} No. The only element of the set is the set that contains the set \{2\}. 
\end{solution}

\begin{problem} (3 points $\times$ 4 subproblems = 12 points)
Section 2.1, Exercise 26, page 132.
\end{problem} 
\begin{solution}
\textbf{a)} Not a power set of a set. Power sets cannot be empty.
\\*
\textbf{b)} This is the power set of S = \{a\}. 
\\*
\textbf{c)} Not a power set of a set. If S = \{$\emptyset$,a\}, then the power set will need to have four elements rather than three.
\\*
\textbf{d)} This is the power set of S = \{a,b\}.
\end{solution}

\begin{problem} (10 points)
Section 2.1, Exercise 28, page 132. 
\textit{Use definitions and justify each step of your argument.}
\end{problem} 
\begin{solution}
If A is a subset of C, that means that every element of A is contained in C. If B is a subset of D, every element of B is contained in D. If you take the Cartesian product of A and B, you will get a set of all pairs (a,b) such that a is an element of A and b is an element of B. If you take the Cartesian product of C and D, you will get a set of all pairs (c,d) such that c is an element of C and d is an element of D. A X B is a subset of C X D because if a is an element of A, it will also be an element of C because A is a susbet of C. The same goes for b because it is an element of B and B is a subset of D. All the pairs in A X B will be accounted for in C X D, making it a subset. 
\\*
A X B = \{(a,b) $|$ a $\in$ A, b$\in$ B\}
\\*
C X D = \{(c,d) $|$ c $\in$ C, d$\in$ D\}
\\*
if a $\in$ A $\rightarrow$ a$\in$ C because A $\subseteq$ C
\\*
if b $\in$ B $\rightarrow$ b$\in$ D because B $\subseteq$ D
\\*
Since a$\in$ C and b$\in$ D, A X B $\subseteq$ C X D
\end{solution}

\begin{problem} (2 points $\times$ 4 subproblems = 8 points) 
Section 2.2, Exercise 4, page 144. 
\end{problem}
\begin{solution}
\textbf{a)} $ A \cup B $ = \{a, b, c, d, e, f, g, h\}
\\*
\textbf{b)} $A \cap B $ = \{a, b, c, d, e\}
\\*
\textbf{c)} A - B = $\emptyset$
\\*
\textbf{d)} B - A = \{f, g, h\}
\end{solution}

\begin{problem} (5 points $\times$ 2 subproblems = 10 points) 
Section 2.2, Exercise 16 c) and d), page 144. 
\textit{Use definitions, and explain each step using definitions and/or laws.}
\end{problem}
\begin{solution}
\textbf{a)} The result of $ A \cap B $ will be a set that contains elements only found in both A and B. That means $ A \cap B \in A $ and $ A \cap B \in B $. Therefore, $ A \cap B \subseteq A $ because we showed that all the elements of $ A \cap B $ will be found in A. 
\\*
\textbf{b)} The result of $ A \cup B $ will be a set that contains all the elements of both A and B. Therefore, $ A \subseteq (A \cup B) $ because we just explained that all the elements of A will be contained in $ A \cup B $ 
\\*
\textbf{c)} The result of A - B will be a set that contains elements found only in A and not in B. Therefore, A - B $\subseteq $ A because there will only be elements contained in A. 
\\*
\textbf{d)} The result of B - A will be a set that contains elements found only in B and not in A. If you do $ A \cap (B - A) $, there will be no elements of A in the B - A portion of the statement. Because of this, the resulting set will have no elements if there are no matching terms in A and B - A. Therefore, $ A \cap (B - A) = \emptyset$ 
\\*
\textbf{e)} The result of B - A will be a set that contains elements found only in B and not in A. However, $ A \cup (B - A) $ means that all elements of A are also included, essentially cancelling out the operation of B - A. Therefore, $ A \cup (B - A) $ = $A \cup B$
\end{solution}

\begin{problem} (5 points $\times$ 2 subproblems = 10 points) 
Section 2.2, Exercise 56 a) and c), page 145. 
\end{problem}
\begin{solution}
\textbf{a)} $\bigcup\limits_{i=1}^{\infty} A_{i}$ = \{1,2,3...\} $\cup$ \{2,3,4...\} $\cup$ ... = $\{x \in$ all positive integers i $| x \geq 1\}$
\\* 
$\bigcap\limits_{i=1}^{\infty} A_{i}$ = \{1,2,3...\} $\cap$ \{2,3,4...\} $\cap$ ... = $\emptyset$
\\*
\textbf{c)} $\bigcup\limits_{i=1}^{\infty} A_{i}$ = (0,1) $\cup$ (0,2) $\cup$... = (0, $\infty$)
\\*
$\bigcap\limits_{i=1}^{\infty} A_{i}$ = (0,1) $\cap$ (0,2) $\cap$... = (0,1) 
\end{solution}

\begin{problem} (3 points $\times$ 4 subproblems = 12 points)
Section 2.3, Exercise 12, page 162. 
\end{problem}
\begin{solution}
\textbf{a)} Yes. If f(a) = f(b), then a - 1 = b - 1. This means a = b, making the function one-to-one.
\\*
\textbf{b)} No. If f(a) - f(b), then $a^2 + 1 = b^2 +1$. Then $a^2 = b^2 = a = \pm b$. This means that a has two possible return values for a, making it not a one-to-one function.
\\*
\textbf{c)} Yes. For all integers, cubing a number will result in a unique number everytime because the sign will not change, unlike when you square a number. Therefore, the function is one-to-one.
\\*
\textbf{d)} No. The ceiling function returns the smallest integer $\geq$ n. This means that different parameters for the function can return the same integer. For example. f(1) = 1 and f(2) = 2, making the function not one-to-one.
\end{solution}

\begin{problem} (3 points $\times$ 2 subproblems = 6 points) 
Section 2.3, Exercise 14 a) and b), page 162.
\end{problem}
\begin{solution}
\textbf{a)} By the definition of the function, m, an integer, will be multiplied by 2, also an integer, and then subtracted by n, another integer. The subtraction operator will allow the function to return any integer, making the range of f equal the the codomain, all integers. Therefore, the function is onto.
\\*
\textbf{b)} The function is taking the difference between two squares. The difference of two squares can only either be odd or a number that is divisible by 4. That means any even number or number that is not divisible by 4 are not in the range of f, making this function not onto. 
\end{solution}

\begin{problem} (2.5 points $\times$ 4 subproblems = 10 points) 
Section 2.3, Exercise 60, page 164. 
\end{problem}
\begin{solution}
\textbf{a)} 1 byte.
\\*
\textbf{b)} 2 bytes.
\\*
\textbf{c)} 63 bytes.
\\*
\textbf{d)} 375 bytes.
\end{solution}

\begin{problem} (15 points) Prove that 
$$ \left\lceil {\left\lceil \displaystyle\frac{x}{2} \right\rceil} \big{/} {2} \right\rceil = 
\left\lceil \displaystyle\frac{x}{4} \right\rceil$$ holds for all real numbers $x$.
\textit{Use the definition of the ceiling function as we discussed in class.}
\end{problem}
\begin{solution}
Let n = $\lceil \lceil x/2 \rceil /2 \rceil $
\\*
By the definition of ceiling functions, $ n - 1 < \lceil x/2 \rceil /2 \leq n$
\\*
This can simplify into, $ 2n-2 < x/2 \leq 2n$ 
\\*
$4n-4 < x \leq 4n$
\\*
\\*
Let m = $\lceil x/4 \rceil$
\\*
By the definition of ceiling functions, $n-1 < x/4 \leq n$
\\*
$4n-4 < x \leq 4n $
\\* \\*
n = m, so $\lceil \lceil x/2 \rceil /2 \rceil $ =  $\lceil x/4 \rceil$ holds for all real numbers x.
\end{solution}
\bigskip

\goodbreak
\checklist

\bigskip
\bigskip
\noindent
\textbf{\LaTeX\ symbols for sets and functions}
\smallskip
\begin{enumerate}
\item
Set of integers that are less than or equal to $n$: $\{x\in \Z\mid x \le n\}$
\item
$x$ is a real number: $x\in \R$
\item 
$x$ is not an integer: $x\not\in \Z$
\item 
Cardinality of set $A$: $|A|$
\item 
Union of set $A$ and set $B$: $A\cup B$
\item
Generalized union: $\bigcup_{i=1}^\infty A_i$
\item
Intersection of set $A$ and set $B$: $A\cap B$
\item 
Generalized intersection: $\bigcap_{i=1}^\infty A_i$
\item
The empty set: $\emptyset$
\item
Set $A$ is a subset of set $B$: $A\subseteq B$
\item
Set $A$ is a proper subset of set $B$: $A\subset B$
\item 
Cartesian product of set $A$ and set $B$: $A \times B$
\item 
Complement of set $A$: $A^C$ or $\overline{A}$
\item
Ellipsis: $\ldots$ or $\cdots$
\item
Ceiling function: $\lceil 3.14 \rceil = 4$
\item 
Floor function: $\lfloor 3.14 \rfloor = 3$
\item 
Square root: $\sqrt{b^2-4ac}$
\end{enumerate}

\end{document}
