\documentclass{article}
\usepackage{amsmath,amssymb,amsthm,latexsym,paralist}

\theoremstyle{definition}
\newtheorem{problem}{Problem}
\newtheorem*{solution}{Solution}
\newtheorem*{resources}{Resources}

\newcommand{\name}[2]{\noindent\textbf{Name: #1}\hfill \textbf{Section: #2}}
\newcommand{\honor}{\noindent On my honor, as an Aggie, I have neither
  given nor received any unauthorized aid on any portion of the
  academic work included in this assignment. Furthermore, I have
  disclosed all resources (people, books, web sites, etc.) that have
  been used to prepare this homework. \\[2ex]
 \textbf{Electronic signature: \underline{ Andrew Han } }}
 
\newcommand{\checklist}{\noindent\textbf{Checklist:}
\begin{compactitem}[$\Box$] 
\item Did you type in your name and section? 
\item Did you disclose all resources that you have used? \\
(This includes all people, books, websites, etc.\ that you have consulted.)
\item Did you sign that you followed the Aggie Honor Code? 
\item Did you solve all problems? 
\item Did you submit  the .tex and .pdf files of your homework to the correct link on eCampus? 
\end{compactitem}
}

\newcommand{\problemset}[1]{\begin{center}\textbf{Problem Set #1}\end{center}}
\newcommand{\duedate}[1]{\begin{quote}\textbf{Due dates:} Electronic
    submission of \textsl{yourLastName-yourFirstName-hw2.tex} and 
    \textsl{yourLastName-yourFirstName-hw2.pdf} files of this homework is due on
    \textbf{#1} on \texttt{http://ecampus.tamu.edu}. You will see two separate links
    to turn in the .tex file and the .pdf file separately. Please do not archive or compress the files.
    \textbf{If any of the two files are missing, you will likely receive zero points for this 
    homework.}\end{quote} }

\newcommand{\N}{\mathbf{N}}
\newcommand{\R}{\mathbf{R}}
\newcommand{\Z}{\mathbf{Z}}


\begin{document}
\vspace*{-15mm}
\begin{center}
{\large
CSCE 222 [Sections 503, 504] Discrete Structures for Computing\\[.5ex]
Fall 2019 -- Hyunyoung Lee\\}
\end{center}
\problemset{2}
\duedate{Friday, 9/13/2019, before 10:00 p.m.}
\name{Andrew Han}{504}
\begin{resources} Discrete Mathematics and its Applications 8th Edition
\end{resources}
\honor

\bigskip

\noindent
\textbf{*** Please make sure that you are solving the correct problems from 
the \underline{8th Edition} of the Rosen book, not the 7th Edition! ***}

\bigskip

\noindent
Total 100 points.
\smallskip

\begin{problem} (5 points $\times$ 2 = 10 points) Section 1.4, Exercise 36 b) and d), page 58.
\end{problem}
\begin{solution}
\textbf{b)} $ \exists x (0 > x > 5) $
\\*
\textbf{c)} $ \forall x (-5 \geq x \geq -1) $
\end{solution}

\begin{problem} (6 points $\times$ 3 = 18 points) Section 1.5, Exercise 28 f), g), and i), page 71. 
\textsl{Justify your answer or give a counterexample.}
[Grading rubric: For each subproblem, saying true / false correctly is 2 points and justifying 
your answer or giving a counterexample is worth 4 points.]
\end{problem}
\begin{solution}
\textbf{f)} True. Every number can be multiplied by its reciprocal to yield 1, which is the same as dividing by itself.
\\*
\textbf{g)} True. If X is greater than 1, add a neagtive number that is 1 less than X. If X is negative, add a number positive enough to yield 1.
\\*
\textbf{i)} False. The statements inside the quantifiers are not algebraically equivalent. If you use X = 3, and Y = -1, for example, both conditions will not be satisfied.
\end{solution}

\begin{problem} (10 points $\times$ 2 = 20 points) Section 1.6, Exercise 14 c) and d), page 83.
\end{problem}
\begin{solution}
\textbf{c)} Modus Ponens and universal instantiation are used. The argument states that if a movie is produced by John Sayles, then the movie is wonderful. Since John Sayles produced a movie about coal miners, there is a wonderful movie about coal miners using universal instantiation and Modus Ponens.
\textbf{d)} Modus Ponens and universal instantiation are used. The argument states that if anyone goes to France, then they will visit the Louvre. Because there was someone in class that has been to France, there is someone in the class who has visited the Louvre using universal instantiation and Modus Ponens.
\end{solution}

\begin{problem} (10 points) Section 1.7, Exercise 6, page 95.  
\end{problem}
\begin{solution}
An odd number can be defined by n = 2k+1 for any integers n and k. If you multiply an odd number with another, you will get 4k$^{2}$ + 4k + 1. Simpliflying this will give you 2(2k$^{2}$ + 2k) + 1, which is an odd number since 2k$^{2}$ + 2k is just another integer. Therefore, the theorem that the product of two odd numbers is also odd is true.
\end{solution}

\begin{problem} (9 points $\times$ 2 = 18 points) Section 1.7, Exercise 20, page 95.
\end{problem}
\begin{solution}
\textbf{a)} The contrapositive of the statement is that if 3n + 2 is odd, then n is odd for any integer n. As seen in the previous problem, multiplying two odd numbers gives you an odd number. In order for 3n + 2 to be odd, the 3n portion needs to be equal to an odd number because if it were even, adding two would yield an even number rather than an odd number. To make 3n odd, n needs to be an odd number, which proves the theorem by contraposition. 
\\*
\textbf{b)} Seeking a contradiction, the theorem now states that 3n + 2 is even and n is odd for any integer n. In order for 3n + 2 to be even, the 3n portion needs to be even because adding 2 to and odd number would yield another odd number and not an even one. To make 3n an even number, n needs to be an even number because our previous problem proved that multiplying two odds would yield another odd. This is a contradiction to our statement if n is odd, so the original theorem is true by contradiction. 
\end{solution}

\begin{problem} (12 points) Prove by \textit{contradiction} that 
if $n\ge 1$ is a perfect square, then $n+2$ is not a perfect square. 
\end{problem}
\begin{solution}
Seeking a contradiction, the statment now becomes n $\geq$ 1 is a perfect square and n + 2 is a perfect square. If we test this theorem with n = 1, a perfect square, we see that n + 2 = 3, which is not a perfect square. This also happens with the next perfect square, n = 4, which gives us n + 2 = 6, also not a perfect square. This contradicts our theorem, which means that the original theorem was true by contradiction. 
\end{solution}

\begin{problem} (12 points) Prove by \textit{contradiction} that
at least three of any 25 days chosen must fall in the same month 
of the year.
\end{problem}
\begin{solution}
Seeking a contradiction, the statement becomes 25 days are chosen and less than 3 days chosen fall in the same month of the year. If you pick a day in a month and go to the next, this cycle will allow you to go through the entire calendar year twice, and then pick a third day in January (or any month). This means that there will be one month that has more than 2 days picked in the same month, contradicting the statement at the start. Therefore, the original theorem is true by contradiction. 
\end{solution} 

\goodbreak
\checklist
\end{document}
